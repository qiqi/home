Uncertainty quantification of numerical simulations has raised significant
interest in recent years.  One of the main challenges remains the
efficiency in propagating uncertainties from the sources
to the quantities of interest, especially when there are many sources of
uncertainties.  The traditional Monte Carlo methods
converge slowly and are undesirable when the required accuracy is high.
Most modern uncertainty propagation methods such as polynomial chaos and
collocation methods, although extremely efficient, suffer from the so called
``curse of dimensionality''.
The computational resources required for these methods grow exponentially
as the number of uncertainty sources increases.

The aim of this work is to address the challenge of efficiently propagating
uncertainties in numerical simulations with many sources of uncertainties.
Because of the large amount of information that can be obtained from
adjoint solutions, we focus on using adjoint
equations to propagate uncertainties more efficiently.

Unsteady fluid flow simulations are the main application of this work, although
the uncertainty propagation methods we discuss are applicable to other
numerical simulations.  We first discuss how to solve the adjoint equations
for time-dependent fluid flow equations.  We specifically address the challenge
associated with the backward time advance of the adjoint equation,
requiring the solution
of the primal equation in backward order.  Two methods are proposed to
address this challenge.  The first method solves
the adjoint equation forward in time, completely eliminating the need for
storing the solution of the primal equation.  The other method is
a checkpointing algorithm specifically designed for dynamic time-stepping.
The adjoint equation is still solved backward in time, but the present
scheme retrieves the primal solution in reverse order.
This checkpointing method is applied to an incompressible Navier-Stokes
adjoint solver on unstructured mesh.

With the adjoint equation solved, we obtain a linear approximation of
the quantities of interest as functions of the random variables describing
the uncertainty sources in a probabilistic setting.
We use this linear approximation to accelerate
the convergence of the Monte Carlo method in calculating tail probabilities
for estimating margins and risk.  In addition, we developed a
multi-variate interpolation scheme that uses multiple adjoint solutions
to construct an interpolant of the quantities of interest as functions of
the uncertainty sources.  This interpolation scheme converge
exponentially to the true function, thus providing very accurate and efficient
means of propagating of uncertainties and remains accurate independently of
the locations of the available data.

