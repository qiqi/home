\chapter[An Unsteady Adjoint Navier-Stokes Solver]
        {An Unsteady Adjoint Solver for the Navier-Stokes Equations}

\section{Mathematical formulation of the adjoint equations}

This section derives the adjoint equation and the adjoint sensitivity gradient
by linearizing the objective functions and the constraints.  Section 
\ref{s:adj_general} treats the most general scenario,
section \ref{s:adj_unsteady} discusses the case when the constraint is a
time-dependent equation, and section \ref{s:adj_ns} derives the adjoint
equation and sensitivity gradient when the constraint is the unsteady
Navier-Stokes equations, and the objective function is a time-averaged
aerodynamic quantity.

\subsection{General adjoint sensitivity gradient} \label{s:adj_general}

Let $\xi$ in a Hilbert space $\Phi$ be the random variables that describes
the sources of uncertainty.  Define the objective function as
$${\bf J}(v,\xi) \in R.$$
The state $v$ is in a Banach space $\bf V$ and satisfies the constraint
$${\bf G}(v,\xi) = 0,$$
where $\bf G: {\bf V} \times \Phi \rightarrow {\bf V}$ is a general nonlinear
operator.  The constraint uniquely determines a state $v$ for any given control
variable $\xi$.

Let $\xi_0 \in \Phi$ and $v_0 \in {\bf V}$ satisfy the constraint
$${\bf G}(v_0,\xi_0) = 0.$$
Let $\xi \in \Phi$ and $v \in {\bf V}$ also satisfy the constraint, and
both $\delta\xi = \xi - \xi_0$ and $\delta v = v - v_0$ be infinitesimal.
Then $\delta\xi$ and $\delta v$ satisfy the linearized
constraint
$$ {\bf L}\, \delta v + {\bf M} \delta \xi =
   {\bf G}(v_0+\delta v, \xi_0+\delta \xi) - {\bf G}(v_0, \xi_0) = 0 $$
where linear operators ${\bf L}: {\bf V} \rightarrow {\bf V}$ and
${\bf M}: \Phi \rightarrow {\bf V}$ are defined as
$$ {\bf L} = \frac{\partial {\bf G}}{\partial v},\quad
   {\bf M} = \frac{\partial {\bf G}}{\partial \xi} \quad
   \mbox{ at } (v_0, \xi_0). $$
On the other hand, the objective function can be linearized as
$$ {\bf J}(v,\xi) = {\bf J}(v_0,\xi_0)
   + \langle\nabla_{\xi}\, {\bf J},\;\delta\xi\rangle
   + \langle\nabla_v\, {\bf J},\;\delta v\rangle $$
$\nabla_{\xi}\, {\bf J} \in {\bf V}^*$ and $\nabla_v\, {\bf J} \in \Phi$
are Frechet derivatives of $\bf J$ at the linearizing point $(\xi_0,v_0)$.

Now let $\hat{v} \in {\bf V}^*$ satisfy the adjoint equation
$${\bf L}^* \hat{v} = \nabla_v\, {\bf J},$$
where $\bf L^*$ is the adjoint operator of the linear operator $\bf L$, then
$$ \langle\nabla_v\, {\bf J},\;\delta v\rangle
 = \langle {\bf L}^* \hat{v},\;\delta v\rangle
 = \langle \hat{v},\; {\bf L}\, \delta v\rangle
 = -\langle \hat{v},\; {\bf M} \delta \xi\rangle
 = -\langle {\bf M}^* \hat{v},\; \delta \xi\rangle $$
and the objective function can be written as
$$ {\bf J}(v,\xi) = {\bf J}(v_0,\xi_0)
 + \langle\nabla_{\xi}\, {\bf J} - {\bf M}^*\hat{v},\;\delta\xi\rangle \;.$$
Using this adjoint method, the sensitivity gradient of the objective
function becomes
\begin{equation} \label{sensitivity_steady}
  \left.\frac{D {\bf J}}{D\xi}\right|_{\xi_0}
 = \nabla_{\xi}\, {\bf J} - {\bf M}^*\hat{v} \;,
\end{equation}
which can be calculated by one adjoint solution.



\subsection{Adjoint sensitivity gradient for unsteady problems}
\label{s:adj_unsteady}

In unsteady problems, the state variable $v$ has an additional time dimension.
i.e., the space of all possible $v$'s can be written as
$$ {\bf V} = {\mathcal V}^{\,[\,0,T]} \;,$$
where $\mathcal V$ is a Banach space.  We call it the ``spatial'' dimension,
as opposed to the temporal dimension.
We denote $v(t) \in {\mathcal V}$ as the state at time $t$.
The norm of $v$ in $\bf V$ should be defined as the supremum of
the norm of $v(t)$ in $\mathcal V$ over $0\le t\le T$.
The constraint in an unsteady problem is a set of time-dependent ordinary or
partial differential equations
$$ {\bf G}(v,\xi) = \frac{\partial v(t)}{\partial t} + G(v(t),\xi) = 0, \quad
   0 \le t \le T \;.$$
In case of a partial differential equation, $G$ is a spatial operator on
$\mathcal V$, while $\xi$ is a parameter of the operator.  On the other
hand, the objective function in unsteady problems can usually be written as
$$ {\bf J}(v,\xi) = \int_0^T J(v(t),\xi;t) \;dt .$$
In the rest of this chapter, $v(t)$ is simply denoted by the abbreviation
$v$ for short where no confusion could result.  Similarly, $v_{0\,t}$
and $\delta v(t)$ are denoted as $v_0$ and $\delta v$.

Let $\xi_0$ and $v_0$ satisfy the constraint
$$ {\bf G}(v_0,\xi_0) = 0 ,$$
and both $\delta v = v - v_0$ and $\delta \xi = \xi - \xi_0$ are infinitesimal.
Linearization of the constraint $\bf G$ around $\xi_0,v_0$ gives
\begin{equation} \label{constraint_l}
 {\bf G}(v,\xi) - {\bf G}(v_0,\xi_0)
 = \frac{\partial\: \delta v}{\partial t} + L(t) \delta v
                                                + M(t) \delta\xi = 0 \;,
\end{equation}
where spatial linear operators $L(t) : {\mathcal V} \rightarrow {\mathcal V}$
and $M(t) : \Phi \rightarrow {\mathcal V}$ are Frechet derivatives defined as
$$ L(t) = \frac{\partial G(v, \xi; t)}{\partial v},\quad
   M(t) = \frac{\partial G(v, \xi; t)}{\partial \xi} \quad
   \mbox{ at } (v_0, \xi_0) \;.$$
These operators may depend on time.  On the other hand, linearization of the
objective function $\bf J$ gives
\begin{equation} \label{td_obj_diff}
  {\bf J}(v,\xi) = {\bf J}(v_0,\xi_0)
 + \int_0^T \left(\langle \nabla_{\xi}\, J, \delta\xi \rangle
                + \langle \nabla_v J, \delta v \rangle \right) dt \;,
\end{equation}
where $\nabla_{\xi}\,J \in \Phi$ and $\nabla_v\,J \in \mathcal{V}^*$ are
Frechet derivatives of $J$ at $(\xi_0,v_0)$, and the brackets here are
spatial inner products, i.e., inner products between spaces $\mathcal V^*$ and
$\mathcal V$.

Now let $\hat v$ satisfy the adjoint equation
$$ -\frac{\partial {\hat v}}{\partial t} + L^* {\hat v} = \nabla_v\, J \;.$$
Then
$$ \int_0^T \langle \nabla_{v}\, J, \delta v \rangle \; dt
 = - \int_0^T \langle \frac{\partial {\hat v}}{\partial t},
                      \delta v \rangle \; dt
   + \int_0^T \langle L^* {\hat v}, \delta v \rangle \; dt \;, $$
where 
$$ - \int_0^T \langle \frac{\partial {\hat v}}{\partial t},
                      \delta v \rangle \; dt
   = \int_0^T \langle \hat v, \frac{\partial\: {\delta v}}{\partial t}
              \rangle \; dt $$
through integration by parts, and
$$ \int_0^T \langle L^* {\hat v}, \delta v \rangle \; dt
 = \int_0^T \langle {\hat v}, L\, \delta v \rangle \; dt $$
through the definition of adjoint operator.  By combining the above three
equations and incorporating Equation (\ref{constraint_l}), we obtain
$$ \int_0^T \langle \nabla_{v}\, J, \delta v \rangle \; dt
 = \int_0^T \langle \hat v, \frac{\partial\: {\delta v}}{\partial t} +
                            L\, \delta v \rangle \; dt
 =-\int_0^T \langle \hat v, M\, \delta \xi \rangle \; dt
 =-\int_0^T \langle M^* \hat v, \delta \xi \rangle \; dt \;.$$
Incorporating this result into Equation (\ref{td_obj_diff}),
the objective function can be obtained as
$$ {\bf J}(v,\xi) = {\bf J}(v_0,\xi_0)
 + \int_0^T \langle \nabla_{\xi} J - M^* {\hat v}, \delta \xi
            \rangle\; dt \;.$$
Therefore, the sensitivity gradient at $\xi_0$ can be calculated by
\begin{equation} \label{sensitivity_unsteady}
 \left.\frac{D{\bf J}}{D\xi}\right|_{\xi_0} = 
 \int_0^T \left(\nabla_{\xi} J - M_t^* {\hat v}\right) dt \;.
\end{equation}
Calculating the sensitivity gradient using this equation can be done by
adding the contribution from each time step of the adjoint solution.



\subsection{Adjoint analysis for incompressible Navier-Stokes flow}
\label{s:adj_ns}

In this subsection, we look at deriving the adjoint sensitivity gradient
for uncertainty quantification problems related to
incompressible Navier-Stokes flow.  The uncertainties affect
both the boundaries and the flow in the interior.  The former describe
The former describes
uncertain wall roughness, unmodeled object movement / oscillations etc.
The latter describe numerical uncertainties and subgrid model uncertainties.
The objective function is assumed to be a measurable aerodynamic quantity, such
as lift, drag or moment of a solid object in the flowfield, which depends on
the pressure and viscous stress on the boundary of the solid object.
The adjoint equations for Navier-Stokes flows have been derived and solved
by \cite{jameson1988}, \cite{bewley01} and \cite{temam}.

The incompressible unsteady Navier-Stokes and continuity equations with
normalized density are
\[ \begin{split}
    & \frac{\partial v}{\partial t} + v \cdot \nabla v
        - \nabla \cdot (\mu \nabla v) + \nabla p = f(\xi) \\
    & \nabla \cdot v = 0 \;,
\end{split} \]
in the spatial domain $\Omega$ and time period $[0, T]$.
The source term $f(\xi)$ depends on $\xi$, the random variables describing
the sources of uncertainties.
The initial condition is
\[ v = u_0, \quad t = 0 \]
and the boundary condition are dependent on the sources of
uncertainties
$$v = v_b(\xi), \quad x \in \partial \Omega .$$
The objective function is a time-integrated aerodynamic quantity, which depends
on the pressure and viscous stresses at the boundary:
\[  {\bf J}(v,\xi) = \int_0^T J({\bf n}\cdot\tau\,|_{\partial\Omega},\,
                  p\, |_{\partial\Omega}; \,t)\; dt \;,\]
where the viscous stress tensor is $\tau = \mu \nabla v$,
and $\bf n$ is the unit wall-normal.

Let $\xi_0$ be the point of linearization, $v_0$ and $p_0$ be the solution of
the Navier-Stokes equations with boundary condition $v_b(\xi_0)$ and
forcing term $f(\xi_0)$.
We can linearize both the Navier-Stokes equations and the objective function.
By defining $\delta \xi = \xi - \xi_0$, $\delta v = v - v_0$ and
$\delta p = p - p_0$, the linearized incompressible LES equation is
\begin{equation} \begin{split} \label{linear_ns}
    & \frac{\partial\: \delta v}{\partial t} + \mathcal{L}_{v_0}\, \delta v
      + \nabla \delta p = \frac{\partial f}{\partial \xi}\, \delta \xi \\
    & \nabla \cdot \delta v = 0
\end{split} \end{equation}
with boundary condition
$$ \delta v = \frac{\partial v_b}{\partial \xi} \, \delta\xi,
   \quad x \in \partial \Omega ,$$
and initial condition
$$ \delta v = 0, \quad t = 0 .$$
The linearized Navier-Stokes operator ${\mathcal L}_{v_0}$ is defined as
$$  \mathcal{L}_{v_0} \delta v = \delta v \cdot \nabla v_0
        + v_0 \cdot \nabla \delta v - \nabla \cdot (\mu \nabla \delta v) \;.$$
Note that the operator depends on the point of linearization $v_0$.

On the other hand, the linearized objective function is
\begin{equation} \label{linear_obj} \begin{split}
   \delta {\bf J}(v,\xi) &= {\bf J}(v,\xi) - {\bf J}(v_0,\xi_0) \\
 &= \int_0^T \iint_{\partial\Omega}
    \left(a(x,t) \cdot ({\bf n} \cdot \delta \tau)
   + b(x,t)\; \delta p \right) ds\; dt \\
 &+ \int_0^T \iiint_\Omega c(x,t) \cdot \delta v\; dx\; dt \;.
\end{split} \end{equation}
where
\[ \delta\tau = \tau - \tau_0 = \mu_0 \nabla \delta v + \delta\mu \nabla v_0 \]
$a(x,t)$, $b(x,t)$ and $c(x,t)$ are the Frechet derivatives
\[ a = \frac{\partial J}{\partial\, ({\bf n} \cdot \tau)} \;, \qquad
   b = \frac{\partial J}{\partial p}\;, \qquad
   c = \frac{\partial J}{\partial v}\]
$a$ is a vector function at the boundaries, $b$ is a scalar function
at the boundaries, and $c$ is a vector function in the interior.

Define the adjoint variables $\hat v$, $\hat p$ and $\hat \mu$ such that they
satisfy the adjoint equation
\begin{equation} \begin{split} \label{adjoint}
    & - \frac{\partial {\hat v}}{\partial t} + \mathcal{L}_{v_0}^* {\hat v}
      + \nabla {\hat p} = -c \\
    & \nabla \cdot {\hat v} = 0
\end{split} \end{equation}
where ${\mathcal L}_{v_0}^*$ is the adjoint operator of ${\mathcal L}_{v_0}$:
\[ {\mathcal L}_{v_0}^* \hat{v} = \nabla v_0 \cdot {\hat v}
        - v_0 \cdot \nabla {\hat v} - \nabla \cdot (\mu_0 \nabla {\hat v}) \;.\]
The adjoint variables also must satisfy the terminal condition
\[ {\hat v} = 0, \quad t = T \]
and the adjoint boundary condition
\begin{equation} \label{nsadjbc}
   {\hat v} = a - {\bf n} \left(b + a \cdot {\bf n} \right),
   \quad x \in \partial\Omega \;.
\end{equation}
We now show that the sensitivity derivatives of the objective function $\bf J$
with respect to $\xi$ can be calculated using $\xi$ and adjoint variables.

First, the adjoint boundary condition (\ref{nsadjbc}) implies that
\begin{equation} \label{nsadj1}
  \delta p\; {\bf n} \cdot {\hat v} = b\: \delta p 
\end{equation}
and because ${\bf n} \cdot \tau \cdot {\bf n} \equiv 0$ on the boundaries,
(\ref{nsadjbc}) further implies
\begin{equation} \label{nsadj2}
   {\bf n} \cdot (\mu \nabla \delta v) \cdot {\hat v}
 = {\bf n} \cdot \delta \tau \cdot {\hat v}
 = - {\bf n} \cdot \delta \tau \cdot a \;.
\end{equation}
In addition, from the adjoint equation,
\begin{equation} \label{nsadj3}
c \cdot \delta v = - \delta v \cdot
      \left( - \frac{\partial {\hat v}}{\partial t}
            + \mathcal{L}_{v_0}^* {\hat v}
            + \nabla {\hat p} \right)
\end{equation}
and
\begin{equation} \label{nsadj4}
0 = \delta p \; \nabla \cdot {\hat v}
\end{equation}
Incorporating (\ref{nsadj1}), (\ref{nsadj2}), (\ref{nsadj3}) and (\ref{nsadj4})
into the linearized objective function (\ref{linear_obj}), we get
\begin{equation} \label{linear_obj1} \begin{split}
   \delta {\bf J}(v,\xi)
   &= \int_0^T \iint_{\partial\Omega}
      \left({\bf n} \cdot (\mu \nabla \delta v) \cdot {\hat v}
    - \delta p \: {\bf n} \cdot {\hat v} \right) ds\; dt \\
   &- \int_0^T \iiint_\Omega \delta v \cdot
      \left( - \frac{\partial {\hat v}}{\partial t}
            + \mathcal{L}_{v_0}^* {\hat v}
            + \nabla {\hat p} \right) dx\;dt \\
   &+ \int_0^T \iiint_\Omega \delta p \; \nabla \cdot {\hat v}\; dx\;dt \\
\end{split} \end{equation}

Furthermore, using integration by parts, and applying the initial and boundary
conditions for both the linearized Navier-Stokes equations and the adjoint
equation, we get the following four equalities:
\begin{equation} \label{nsadj5}
   \int_0^T {\hat v} \cdot \frac{\partial\, \delta v}{\partial t} dt
 = - \int_0^T \delta v \cdot \frac{\partial {\hat v}}{\partial t} dt
\end{equation}
\begin{equation} \label{nsadj6} \begin{split}
   \iiint_\Omega {\hat v} \cdot {\mathcal L}_{v_0} \delta v\; dx
 &= \iiint_\Omega \delta v \cdot {\mathcal L}_{v_0}^* {\hat v}\; dx \\
 &- \iint_{\partial\Omega}
   {\bf n} \cdot (\mu \nabla \delta v) \cdot {\hat v}\; ds \\
 &+ \iint_{\partial\Omega}
   {\bf n} \cdot (\mu \nabla {\hat v} - v_w{\hat v}) \cdot \delta v\; ds
\end{split} \end{equation}
\begin{equation} \label{nsadj7}
   \iiint_\Omega {\hat v} \cdot \nabla \delta p\; dx
 = - \iiint_\Omega \delta p\; \nabla \cdot {\hat v}\; dx
 + \iint_{\partial\Omega} \delta p\; {\bf n}\cdot {\hat v} \; ds
\end{equation}
\begin{equation} \label{nsadj8}
   \iiint_\Omega {\hat p}\; \nabla \cdot \delta v\; dx
 = - \iiint_\Omega \delta v \cdot \nabla {\hat p}\; dx
 + \iint_{\partial\Omega} {\hat p}\; {\bf n}\cdot \delta v\; ds \;.
\end{equation}
Incorporating these four equations
(\ref{nsadj1}), (\ref{nsadj2}), (\ref{nsadj3}) and (\ref{nsadj4}) into
(\ref{linear_obj1}), we get
\[ \begin{split}
   \delta {\bf J}(v,\xi) =&
   \int_0^T \iint_{\partial\Omega} ( {\bf n} \cdot (\mu \nabla {\hat v}) -
   {\bf n} \cdot v_w\; {\hat v} - {\hat p}\; {\bf n} )\cdot \delta v\; ds \\
   &- \int_0^T \iiint_\Omega {\hat v} \cdot
      \left( \frac{\partial \,\delta v}{\partial t}
        + \mathcal{L}_{v_0} \delta v
        + \nabla \delta p \right) dx\;dt \\
   &+ \int_0^T \iiint_\Omega {\hat p} \; \nabla \cdot \delta v\; dx\;dt \\
   =& \int_0^T \iint_{\partial\Omega} ( {\bf n} \cdot (\mu \nabla {\hat v}) -
   {\bf n} \cdot v_w\; {\hat v} - {\hat p}\; {\bf n} )\cdot
   \frac{\partial v_b}{\partial \xi} \:\delta \xi \; ds  \\
   &- \int_0^T \iiint_\Omega {\hat v} \cdot \frac{\partial f}{\partial\xi} \:
     \delta \xi\; dx\;dt \;.
\end{split} \]
Therefore, the sensitivity gradient is
\begin{equation} \begin{split}
  \frac{D {\bf J}}{D \xi} &= 
   \int_0^T \iint_{\partial\Omega} ( {\bf n} \cdot (\mu \nabla {\hat v}) -
   {\bf n} \cdot v_w\; {\hat v} - {\hat p}\; {\bf n} )\cdot
   \frac{\partial v_b}{\partial \xi} \; ds  \\
   &- \int_0^T \iiint_\Omega {\hat v} \cdot \frac{\partial f}{\partial\xi}
     \; dx\;dt \;.
\end{split} \end{equation}
It can be calculated by using the adjoint solution at each time step.
This is a key motivation for obtaining the adjoint solution in the first place.



% ======================================================================= %
%
% ======================================================================= %

\section{The adjoint Navier-Stokes solver}

This section describes the numerical scheme used to solve the adjoint
Navier-Stokes equations.  In designing the numerical scheme, we follow the
``discrete adjoint'' approach.  The numerical scheme we obtain by this
approach is not only a consistent discretization of the continuous
adjoint Navier-Stokes equations, it is also the direct discrete adjoint
of the numerical scheme for solving the unsteady Navier-Stokes equations.
This approach has an important advantage\footnote{
The discrete adjoint also have disadvantages compared to the
continuous adjoint, such as its dependency on the primal scheme, and generally
greater complexity.}:
the sensitivity gradient we obtained
is not affected by discretization error of either the Navier-Stokes equation
or the adjoint Navier-Stokes equations.  Since the discrete adjoint scheme is
dependent on the scheme for solving the Navier-Stokes equations, we start by
describing CDP, our unsteady Navier-Stokes solver on an unstructured hybrid
mesh.

\subsection{The scheme for solving the unsteady Navier-Stokes equations}

The 3-D unsteady Navier-Stokes solver is called CDP, and is
developed and maintained by Dr. Frank Ham of the Center for Turbulence
Research \cite[]{cdp1}.  The solver uses a fractional-step method
to enforce the divergence-free condition.
In the most recent version of the solver, an implicit BDF-2
time integration scheme \cite[]{bdf2} is used for time discretization.
The spatial discretization is a node based finite volume method.
Velocity and pressure are stored in the dual-volume corresponding to each node.
They are denoted as $u$ without subscript, and $p$, respectively.
A separate velocity field is stored on the surfaces of the dual-volumes,
precisely corresponding to the face edges, to enforce the divergence-free
condition.  This auxiliary velocity field
is denoted as $u_m$.  The Simplex Superposition scheme \cite[]{ss}
recently developed by Dr. Ham is used for spatial
discretization of the gradient, divergence, Laplacian and convection operators.
The scheme is second-order accurate both in space and time for arbitrarily
shaped elements.

A mathematical description of the discretization scheme is
\begin{equation} \begin{split} \label{pred-corr}
   u_m^{k*} = 2 u_m^{k-1} - u_m^{k-2} \\
   u^{k-\frac23} = \frac43 u^{k-1} - \frac13 u^{k-2} \\
   \frac32 \frac{u^{k*} - u^{k-\frac23}}{\Delta t} =
 - u_m^{k*} \cdot \nabla u^{k*} + \nabla \cdot \mu \nabla u^{k*}
 - \nabla p^{k-1}  \\
   u^{k-} = u^{k*} + \frac{2 \Delta t}{3} \nabla p^{k-1}  \\
   \Delta p^k = \frac3{2 \Delta t} \nabla \cdot u^{k-}  \\
   u^k = u^{k-} - \frac{2 \Delta t}{3} \nabla p^k  \\
   u_m^k = P_F\, u^{k-} - \frac{2\Delta t}{3} \nabla_F\, p^k \;.
\end{split} \end{equation}
In this formula, the first three lines are the prediction step, including
the BDF-2 scheme (the second and the third lines); the last
four lines are the pressure correction step.  $u^{k-1}$ is the nodal velocity at
time step $k-1$.  $u_m^{k-1}$ is the auxiliary divergence-free velocity
stored on the interface of the dual-volumes at time step $k-1$.
$u_m^{k*}$ is the extrapolated auxiliary velocity field based on $u_m$
at time steps $k-1$ and $k-2$, and is used as the convecting velocity
in the next prediction step.
$u^{k-\frac23}$ is an extrapolated nodal velocity field from time step $k-1$
and $k-2$, and is used by the BDF-2 time integration scheme to achieve
its formal second-order time accuracy.  $u^{k*}$ is the nodal velocity
at time step $k$ predicted by the prediction step.  It is obtained by
time-integrating the full Navier-Stokes equations using the BDF-2 scheme,
using the pressure field $p^{k-1}$ at time step $k-1$ instead of the
pressure field at time step $k$.
$u^k$, the final nodal velocity field, is obtained after the correction step.
After $u^{k*}$ is calculated, we calculate $u^{k-}$ by subtracting the effect
of the pressure field at time step $k-1$ from $u^{k*}$.
The pressure $p^k$ at time step $k$ is then calculated from the divergence of
$u^{k-}$.  The nodal gradient of the new pressure field is then added to
$u^{k-}$ to obtain $u^k$, the nodal velocity at time step $k$.
The new auxiliary velocity field, $u_m^k$, is obtained by transferring $u^{k-}$
to the dual-cell interfaces, then add the interfacial gradient of the new
pressure field $p^k$.  The transfer operator $P_F$ and interfacial gradient
operator $\nabla_F$ are designed so that the divergence of $u_m^k$ is exactly
zero.  The $k$th time step is completed when $p^k$, $u^k$ and $u_m^k$ are
obtained.

Six discrete spatial operators are used in the scheme:
the convection operator $u_m \cdot \nabla$, the nodal divergence operator
$\nabla \cdot$, the nodal gradient operator $\nabla$, the nodal Laplacian
operator $\Delta$, the nodal to facial transfer operator $P_F$, and
the nodal to facial gradient operator $\nabla_F$.
In the most recent version of CDP, these discrete operators are constructed
with the Simplex Superposition principle \cite[]{ss}.  These operators satisfy
$\nabla_F \cdot \nabla_F = \Delta$ and $\nabla_F \cdot P_F = \nabla \cdot$,
ensuring the discrete divergence-free condition of the facial velocity $u_m$,
i.e., $\nabla_F \cdot u_m = 0$.  The Simplex-Superposition based schemes have
excellent memetic properties \cite[]{ss}, in the sense that the discrete
operators retain many benign properties of their continuous counterparts,
such as discrete conservation of kinetic energy \cite[]{kec}.
In addition, due to their memetic properties,
the Laplacian operator is discretely self-adjoint,
while the nodal divergence operator and the nodal gradient operator are
discretely adjoint to each other.
These properties makes construction of the discrete adjoint scheme easier.


\subsection{The homogeneous adjoint scheme}

A homogeneous adjoint equation is an adjoint equation with zero source term
and trivial boundary conditions.  It corresponds to the continuous adjoint
equation (\ref{adjoint}) and boundary condition (\ref{nsadjbc}) with
$a = b = c = 0$.
The following equations describe the discrete adjoint version of
(\ref{pred-corr}) that solves the homogeneous adjoint equation.
\begin{equation} \begin{split} \label{pred-corr-adj}
  \Delta q^k = \frac{\hat p^k + \nabla \cdot \hat u^k}{\Delta t} +
                \nabla_F \cdot \hat u_m^{k*} \\
   \hat u^{k-} = \hat u^k + \Delta t \left( P_N\,\hat u_m^{k*} -
                                            \nabla q^k \right) \\
   \frac 32 \frac{\hat u^{k-\frac23} - \hat u^{k-}}{\Delta t} =
   u_m^{k*} \cdot \nabla \hat u^{k-\frac23}
 + \nabla \cdot \mu \nabla \hat u^{k-\frac23} \\
   \hat u^{k-1} = \frac43 \hat u^{k-\frac23} - \frac13 \hat u^{k+\frac13} \\
   \hat u_m^{k} = -\frac23\, \nabla u^{k*} \cdot \hat u^{k-\frac23} \\
   \hat p^{k-1} = \nabla \cdot \left(\hat u^{k-\frac23} - \hat u^{k-}\right) \\
   \hat u_m^{k-1*} = 2 \hat u_m^{k} - \hat u_m^{k+1}
\end{split} \end{equation}
A brief derivation of this scheme is presented in Appendix A.
One can show that this scheme is a consistent (though unusual) discretization
of the homogeneous continuous adjoint equation, i.e., equation (\ref{adjoint}),
with zero source term $c$.
Therefore, as the mesh is refined, the solution using this scheme converges to
the continuous adjoint solution.  In addition, this scheme is the discrete
adjoint of the scheme (\ref{pred-corr}).
If we write the discrete tangent linear Navier-Stokes
equations as a single matrix equation whose solution consists of its solution
at all grid points and all time steps, our discrete adjoint scheme
(\ref{pred-corr-adj}) would be
equivalent to solving the transpose of the huge matrix equation.  For this
reason, the adjoint solution of this scheme and the Navier-Stokes solution using
scheme (\ref{pred-corr}) satisfy the discrete adjoint condition\footnote{
A rigorous proof of this discrete adjoint condition is presented in Appendix A.
}:
\[\begin{split}
 & \sum \left( \delta u^k \cdot \hat u^k -
   \frac13\, \delta u^{k-1} \cdot \hat u^{k+\frac13} \right) dV + \\
 & \sum \Delta t \left( \delta u_m^k \cdot \hat u_m^{k*} -
                  \delta u_m^{k-1} \cdot \hat u_m^{k+1} +
                  \frac23\, \delta p^k \hat p^k \right) dV \\
=& \sum \left( \delta u^{k-1} \cdot \hat u^{k-1} -
   \frac13\, \delta u^{k-2} \cdot \hat u^{k-\frac23} \right) dV + \\
 & \sum \Delta t \left( \delta u_m^{k-1} \cdot \hat u_m^{k-1*} -
                   \delta u_m^{k-2} \cdot \hat u_m^k
                + \frac23\, \delta p^{k-1} \hat p^{k-1} \right) dV \;.
\end{split} \]
Or equivalently,
\[\begin{split}
 & \sum \left( \delta u^k \cdot \hat u^k -
   \frac13\, \delta u^{k-1} \cdot \hat u^{k+\frac13} \right) dV + \\
 & \sum \Delta t \left( \delta u_m^k \cdot \hat u_m^{k*} -
                  \delta u_m^{k-1} \cdot \hat u_m^{k+1} +
                  \frac23\, \delta p^k \hat p^k \right) dV 
= \mbox{Constant for all } k \;.
\end{split} \]
In this discrete adjoint condition, $\sum dV$ represents a numerical
integration of a scalar field over the entire volume.  Specifically,
\begin{equation} \label{empty_summation}
\sum \phi\; dV = \sum \phi_i dV_i ,
\end{equation}
where the summation is over all the grid points in the mesh, $\phi_i$ is the
value of the scalar field on the $i$th mesh point, and $dV_i$ is the volume
of the corresponding dual-control volume.  This notation is also used to
describe the inhomogeneous discrete adjoint condition in the next subsection.
$\Delta t$ is the size of the time step between the $k-1$st and the
$k$th time steps.
This discrete adjoint condition is satisfied with any mesh resolution and
time step.

The adjoint equation evolves backward in time; its terminal condition is
set at the last time step of the Navier-Stokes time integration, and each
evaluation of the adjoint scheme (\ref{pred-corr-adj}) brings the adjoint
solution one time step backward.  Note that the primal intermediate facial and
nodal velocities $u_m^{k*}$ and $u^{k*}$ appear in (\ref{pred-corr-adj}), which
is consistent with the fact that the continuous form of the adjoint equation
(\ref{adjoint}) contains the primal velocity $u$.  This fact suggests that
applying the adjoint scheme at the $k$th time step requires information from the
Navier-Stokes solution at the same time step.  For this reason, solving the
adjoint equation for each time step involves some preparation before
performing the calculation in (\ref{pred-corr-adj}):
\begin{enumerate}
\item Obtain the Navier-Stokes state variables $u^{k-1}$, $u^{k-2}$,
      $u_m^{k-1}$, $u_m^{k-2}$ and $p^{k-1}$.
\item Solve the prediction part (first three lines) of the Navier-Stokes step
      (\ref{pred-corr}) for $u^{k*}$ and $u_,^{k*}$.
\item Retrograde the adjoint solution using the adjoint scheme
      (\ref{pred-corr-adj}).
\end{enumerate}
Since the adjoint equation is calculated retrograde in time, the Navier-Stokes
state variables are needed in a reverse-time order.  The dynamic checkpointing
scheme described in Chapter 3 is applied to fulfill this requirement.
Using our dynamic checkpointing scheme, the adjoint equation can be solved at
a computational cost of around 4 to 8 times that of the Navier-Stokes equations,
with moderate increase in memory usage.


\subsection{The inhomogeneous adjoint scheme and sensitivity gradient}

In the previous section, we discussed the scheme for solving homogeneous adjoint
equation, i.e., the adjoint equation with a zero source term and trivial
boundary conditions.  In practice, most objective functions are
time integrals of physical quantities in the flowfield or on the boundary.
Consequently, the adjoint equations for these objective functions has either
a nonzero source term or nontrivial boundary condition.
This section focuses on these inhomogeneous adjoint equations.
We discuss how to construct the source term based on the specific form of
the objective function, and how to calculate the sensitivity gradients using
the solution of the adjoint equation.

In general, the objective function depends on the velocity field and the
pressure field.  Denote the objective function as $\bf J$, and let
\[ J_{u\,i}^k = \frac{\partial {\bf J}}{\partial u_i^k} \;,\quad
   J_{p\,i}^k = \frac{\partial {\bf J}}{\partial p_i^k} \]
the derivatives of the objective function with respect to the velocity and
pressure at every grid point and every time step.
The general form of the objective function can then be linearized as
\[ \delta {\bf J} = \sum_{k=0}^{M_T} \Delta t
            \sum \left(J_u^k \cdot \delta u^k + J_p^k\,\delta p^k\right) dV ,\]
where $M_T$ is the last time step in the problem considered, and the
inner summation is as defined in Equation (\ref{empty_summation}).
Similarly, a general assumption with the uncertainties is that they add a source
term to the momentum equation anywhere and anytime.  In other words,
the uncertainties lead to the following modification of the velocity field
\[ u^k = u^k + \Delta t\; f^k(\xi) \]
at the end of every time step of the Navier-Stokes equations.
Its linearized form is
\begin{equation} \label{source_linearns}
\delta u^k = \delta u^k + \Delta t\;
             \frac{\partial f^k}{\partial \xi}\;\delta\xi .
\end{equation}
The initial condition can also be uncertain, and is a function of $\xi$.
The linearized initial condition is
\[ \begin{split}
& \delta u^0 = \frac{\partial u^0}{\partial \xi}\;\delta\xi \\
& \delta u_m^0 = \frac{\partial u_m^0}{\partial \xi}\;\delta\xi \;.
\end{split} \]
The form of the inhomogeneous adjoint equation depends on the linearized
objective function, specifically on $J_u$ and $J_p$.  On the other hand,
the formula for computing the sensitivity gradient from the adjoint solution
depends on how the sources of uncertainties influence the system, specifically
on $\dfrac{\partial f^k}{\partial \xi}$.

We solve the inhomogeneous adjoint equation with the same scheme as
(\ref{pred-corr-adj}), except at the end of each time step, we let
\begin{equation} \begin{split} \label{source_adjns}
& {\hat u}^{k-1} = {\hat u}^{k-1} + \Delta t\; J_u^{k-1} \\
& {\hat p}^{k-1} = {\hat p}^{k-1} + \frac32\; J_p^{k-1} \;,
\end{split} \end{equation}
where the $\frac32$ in the adjoint pressure update is due to the BDF-2 scheme.
This addition of a source term in the linearized Navier-Stokes equations as
in (\ref{source_linearns}), and in the adjoint Navier-Stokes equations as in
(\ref{source_adjns}), both modifying the discrete adjoint condition from its
homogeneous form.  The inhomogeneous discrete adjoint condition is
\[\begin{split}
 & \sum \left( \delta u^k \cdot \hat u^k -
   \frac13\, \delta u^{k-1} \cdot \hat u^{k+\frac13} \right) dV + \\
 & \sum \Delta t \left( \delta u_m^k \cdot \hat u_m^{k*} -
                  \delta u_m^{k-1} \cdot \hat u_m^{k+1} +
                  \frac23\, \delta p^k \hat p^k \right) dV - 
 \Delta t \sum \left(\frac{\partial f^k}{\partial \xi} \cdot {\hat u}^k
                 \;\delta\xi\right) dV \\
=& \sum \left( \delta u^{k-1} \cdot \hat u^{k-1} -
   \frac13\, \delta u^{k-2} \cdot \hat u^{k-\frac23} \right) dV + \\
 & \sum \Delta t \left( \delta u_m^{k-1} \cdot \hat u_m^{k-1*} -
                   \delta u_m^{k-2} \cdot \hat u_m^k
                + \frac23\, \delta p^{k-1} \hat p^{k-1} \right) dV - \\
 & \Delta t \sum \left( \delta u^{k-1} \cdot J_u^{k-1}
                      + \delta p^{k-1} J_p^{k-1} \right) dV \;,
\end{split} \]
where the summation is as defined in Equation (\ref{empty_summation}).
Applying this equality for $k=1,2,\ldots,M_T$, we get
\begin{equation} \label{inhom_adj_cond} \begin{split}
 & \sum \left( \delta u^{M_T} \cdot \hat u^{M_T} -
   \frac13\, \delta u^{M_T-1} \cdot \hat u^{M_T+\frac13} \right) dV + \\
 & \sum \Delta t \left( \delta u_m^{M_T} \cdot \hat u_m^{M_T*} -
                  \delta u_m^{M_T-1} \cdot \hat u_m^{M_T+1} +
                  \frac23\, \delta p^{M_T} \hat p^{M_T} \right) dV + \\
 & \sum_{k=0}^{M_T-1} \Delta t
 \sum \left( \delta u^{k} \cdot J_u^{k}
           + \delta p^{k} J_p^{k} \right) dV \\
=& \sum \left( \delta u^0 \cdot \hat u^0 -
   \frac13\, \delta u^{-1} \cdot \hat u^{\frac13} \right) dV + \\
 & \sum \Delta t \left( \delta u_m^0 \cdot \hat u_m^{0*} -
                  \delta u_m^{-1} \cdot \hat u_m^{1} +
                  \frac23\, \delta p^0 \hat p^0 \right) dV + \\
 & \sum_{k=1}^{M_T} \Delta t
 \sum \left(\frac{\partial f^k}{\partial \xi} \cdot {\hat u}^k
                 \;\delta\xi\right) dV \;.
\end{split}\end{equation}
When the simulation starts, $u^{-1}$, $u_m^{-1}$ and $p^0$ are set to 0.
So there are no uncertainties in these quantities.
Also, we set the adjoint terminal condition as
\begin{equation} \begin{split} \label{source_adjns}
& {\hat u}^{M_T} = \Delta t\; J_u^{k-1} \\
& {\hat p}^{M_T} = \frac32\; J_p^{k-1} \\
& {\hat u}^{M_T+\frac13} = 0 \\
& {\hat u_m}^{M_T} = 0 \\
& {\hat u_m}^{M_T+1} = 0  \;.
\end{split} \end{equation}
The inhomogeneous discrete adjoint condition (\ref{inhom_adj_cond}) becomes
\[\begin{split}
 & \sum_{k=0}^{M_T} \Delta t
 \sum \left( \delta u^{k} \cdot J_u^{k}
           + \delta p^{k} J_p^{k} \right) dV \\
=& \sum \left( \delta u^0 \cdot \hat u^0 +
 \Delta t\; \delta u_m^0 \cdot \hat u_m^{0*} \right) dV +
 \sum_{k=1}^{M_T} \Delta t
 \sum \left(\frac{\partial f^k}{\partial \xi} \cdot {\hat u}^k
                 \;\delta\xi\right) dV  \;.
\end{split}\]
In other words,
\[ \delta {\bf J}
= \sum \left( \frac{\partial u^0}{\partial\xi} \cdot \hat u^0 +
 \Delta t\; \frac{\partial u_m^0}{\partial\xi} \cdot \hat u_m^{0*} \right)
  \delta \xi\;dV +
 \sum_{k=1}^{M_T} \Delta t
 \sum \left(\frac{\partial f^k}{\partial \xi} \cdot {\hat u}^k
                 \;\delta\xi\right) dV \;.\]
Therefore, the sensitivity gradient can be calculated by
\begin{equation} \label{sensitivity_discrete}
\frac{D {\bf J}}{D \xi}
= \sum \left( \frac{\partial u^0}{\partial\xi} \cdot \hat u^0 +
 \Delta t\; \frac{\partial u_m^0}{\partial\xi} \cdot \hat u_m^{0*} \right)
 \;dV +
 \sum_{k=1}^{M_T} \Delta t
 \sum \left(\frac{\partial f^k}{\partial \xi} \cdot {\hat u}^k \right) dV \;.
\end{equation}
We note that the adjoint solution at all time steps is used in calculating
the sensitivity gradient.  After the adjoint solution is computed at
each time step $k$,
$ \sum \left(\frac{\partial f^k}{\partial \xi} \cdot {\hat u}^k \right) dV $
is calculated and added to the sensitivity gradient.  The full
sensitivity gradient can be computed only after the adjoint solution reaches
time step $0$.



\section{Numerical example of an adjoint solution} \label{s:cylinder_adj}

The last section discussed how to solve the adjoint Navier-Stokes
equations for general objective functions.  This section gives an example of
the process.  We consider a Newtonian incompressible fluid
flowing past an infinitely long circular cylinder.  The Reynolds number
with respect to the far-field flow velocity and the cylinder diameter is 100.
At this Reynolds number, the flowfield is 2-D, unsteady and periodic
in time \cite[]{fey:1547}.

The incompressible Navier-Stokes equations were solved in a
2-D domain of size 60 cylinder diameters (flow direction) by 80
cylinder diameters (crossflow direction).  The domain was discretized with
approximately 10,000 unstructured quadrangle mesh elements.
By taking advantage of
the unstructured grid, most of the mesh points were concentrated
around the cylinder
and in a band of length 10 cylinder diameters downstream of the cylinder.
Figure \ref{mesh1} shows the unstructured mesh used.
\begin{figure}[htb!] \center 
\includegraphics[width=2.34in]{output_adj/mesh1.png}
\includegraphics[width=3.0in]{output_adj/mesh2.png}
\caption{The mesh used for calculating the flowfield at Reynolds number 100.
The left picture shows the entire computational domain of $60 \times 80$;
the right picture zooms in to a small region around the cylinder.}
\label{mesh1}
\end{figure}
The flow velocity on the left
boundary of the domain was set to the freestream velocity, the upper and lower
boundaries of the domain were set to periodic, and an convective outlet
boundary condition
was used at the downstream boundary.  An algebraic multigrid solver was used
to solve the Poisson equation in the corrector steps.  We used a fixed time
step size, $\Delta t = 0.1$, in this calculation.  The corresponding CFL
number was approximately $3$ in the region near the cylinder.
All physical quantities were
normalized with respect to the fluid density, the freestream velocity and
the cylinder diameter.

The uncertainty in the problem we consider comes from small rotational
oscillations of the cylinder in the flow.
The specific form of the oscillation is
unknown, and the rate of rotation $\omega$ is modeled as a random time process.
A no-slip boundary condition is used on the wall, which is
\[ v_x = \omega y, \quad v_y = -\omega x. \]
 
Oscillatory rotation of the cylinder has been shown to
have significant effect on the drag coefficient \cite[]{rotary_control}.
The objective of this example is to analyze the effect of the small
random rotational oscillations on the drag coefficient.  Specifically,
we want to obtain the probability distribution of the time-averaged drag
coefficient of the cylinder from the random rotation.
We describe this process as `'propagating'' the uncertainties from
the sources, the random rotation in this example,
to the objective quantity, the time-averaged drag coefficient $\bar{c_d}$.

The objective function is
\[ {\bf J} = \overline{c_d} = \frac{1}{T} \int_0^T c_d\: dt \;,\]
where $c_d$ is the instantaneous drag coefficient,
\[ c_d = \frac{D(t)}{\frac12 \rho v_\infty^2 d} \;.\]
$D(t)$ is the instantaneous drag on the cylinder per spanwise length,
$\rho$ is fluid density, $v_\infty$ is freestream velocity and $d$ is the
cylinder diameter.  Since all quantities are normalized with $\rho$,
$v_\infty$ and $d$, $\rho v_\infty^2 d = 1$, and
\[ c_d = 2 D(t) \;.\]
The instantaneous unit-span drag $D(t)$ consists of pressure drag $D_p$ and
viscous drag $D_v$, which are
\[ D_p = -\oint p {\bf n} \cdot {\bf e_x} ds \]
\[ D_v = \oint {\bf n} \cdot \tau \cdot {\bf e_x} ds \]
and
\[ c_d = 2 \oint ({\bf n} \cdot \tau - p {\bf n}) \cdot {\bf e_x} \; ds \]
where $\bf n$ is wall-normal unit vector,
$\mu$ is viscosity, ${\bf e_x} = (1,0,0)$ is the
unit vector in the $x$-direction, and the integration is on the circumference
of the cylinder cross-section.

A numerical approximation of the time-averaged drag coefficient is
\begin{equation} \label{discrete_meancd}
  {\bf J} = \frac{2}{T} \sum_{k=1}^T \Delta t
         \sum_{i \in sfc}
         ({\bf n}_i \cdot \tau_i - p_i {\bf n}_i) \cdot {\bf e_x} ds_i\; .
\end{equation}
In the formula, $sfc$ represents mesh nodes on the wall boundary of
the cylinder, $p_i$ is the pressure, and $\tau_i$ is the
viscous stress at node $i$, which can be directly calculated from the velocity
of its neighboring nodes:
\[ \tau_i = \frac{1}{Re} \nabla v |_i
 = \frac{1}{Re} \sum_{j\in nbr(i)} \kappa_{i\,j} v_{j} \;, \]
where $\kappa_{i\,j}$ is the Simplex-Superposition discretization of
the gradient operator, its specific value depends on the cell types and
geometry.  We note that this objective function is linear with
respect to the velocity and pressure.  The source term of the
adjoint equation corresponding to this objective function is
\begin{equation} \begin{split} \label{drag_adjsrc}
 J_{v\,i} &= \sum_{j \in sfc \cup nbr(i)} {\bf n}_j \cdot \kappa_{j\,i}
             {\bf e_x} \\
 J_{p\,i} &= \begin{cases} {\bf n}_i \cdot {\bf e_x} & i \in sfc \\
                           0 & i \notin sfc \quad.\end{cases}
\end{split} \end{equation}

\begin{figure}[htb!] \center
\includegraphics[width=2.05in]{output_adj/u1.png}
\includegraphics[width=2.05in]{output_adj/adj1.png}
\includegraphics[width=2.05in]{output_adj/u2.png}
\includegraphics[width=2.05in]{output_adj/adj2.png}
\includegraphics[width=2.05in]{output_adj/u3.png}
\includegraphics[width=2.05in]{output_adj/adj3.png}
\includegraphics[width=2.05in]{output_adj/u4.png}
\includegraphics[width=2.05in]{output_adj/adj4.png}
\caption{Flow and adjoint solutions at $t=125.0, 126.5, 128.0, 129.5$
(upper-left, upper-right, lower-left, lower-right)}
\label{solutions}
\end{figure}

We use the numerical schemes (\ref{pred-corr-adj}) and (\ref{source_adjns})
with the source term (\ref{drag_adjsrc}) to solve the inhomogeneous adjoint
equation, so that the corresponding objective function is the discretized
time-averaged drag coefficient (\ref{discrete_meancd}).
Several snapshots of the adjoint solution are depicted in
Figure \ref{solutions}.  The left side of the figure shows the streamwise
velocity $u_x$ at four time instances;
the right side shows the corresponding adjoint streamwise velocity
${\hat u}_x$.  These adjoint fields reveals the sensitivity derivative
of the objective function with respect to changes in the flowfield.
In these plots, red is positive values of ${\hat u}_x$, indicating that
a positive change in the streamwise velocity would increase the drag of
the cylinder.  Such regions include the upstream of the cylinder, where
an increase in $u_x$ would generate more skin friction drag,
and unsteady bands in the shear layers on the sides of the cylinder,
where an increase in $u_x$
would temporarily widen the wake, creating more pressure drag on the cylinder.
In contrast, blue is negative values of ${\hat u}_x$, indicating that a
positive change in the streamwise velocity would reduce the drag of the
cylinder.  These regions are unsteady bands in the shear layers where
an increase in $u_x$ would temporarily narrow the wake, decreasing the pressure
drag.

By integrating the adjoint solution using (\ref{sensitivity_discrete}), we
obtain the sensitivity gradient of the objective function with respect to
the random variables describing the sources of uncertainty.  In this case,
it is the sensitivity gradient of the time-averaged drag coefficient with
respect to the rate of rotation $\omega$.  Since $\omega$ is a random time
process, the sensitivity gradient obtained is a function of time.  This
function
${\hat \omega} = \left. \dfrac{\partial {\bf J}}{\partial \omega} \right|_t$
is plotted in Figure \ref{mean}, along with the time history of the
instantaneous lift and drag coefficients on the cylinder.

\begin{figure}[htb!] \center
\includegraphics[width=5in]{output_cdp003/mean.png}
\caption{The time history of $c_l$, $c_d$ and
$\hat{\omega}= \frac{\partial \bf J}{\partial \omega}$ of flow past a
non-rotating circular cylinder at $Re=100$.  The objective function for the
adjoint equation is the time-integrated drag on the cylinder.
The time averaged drag coefficient $1.3345$ is plotted as a dotted line in
the $c_d$ plot.}
 \label{mean}
\end{figure}

This example completes our discussion of how to solve the adjoint equation for
the unsteady incompressible Navier-Stokes equations, and how to calculate the
sensitivity gradient from the adjoint solution.  The next two chapters
discuss how to use this sensitivity gradient in uncertainty quantification
problems.
