\chapter{Conclusion and Future Work}

The primary goals of this work are to develop an efficient method of solving
the adjoint equation for unsteady fluid flow equations, and to use the adjoint
solution to efficiently propagate uncertainties in computational fluid dynamics
simulations.  Both goals have been achieved for incompressible
Navier-Stokes flow simulations in complex geometry using unstructured grids.
Two ways of efficiently solving the adjoint equation for time-dependent
problems have been demonstrated:
the Monte Carlo forward-time adjoint solver, and
the dynamic checkpointing method.  We applied the dynamic checkpointing
method in an unstructured incompressible Navier-Stokes solver, and integrated
it into CDP, the incompressible flow solver used at the Center for Turbulence
Research at Stanford University.
We also developed two approaches for using the adjoint
solution in propagating uncertainties.  One approach is using the adjoint
sensitivity gradient to accelerate Monte Carlo methods in quantification
of risk.  The other approach is a multi-variate interpolation scheme.
The scheme uses the gradient information on grid points to improve accuracy.
It works for arbitrary scattered grid points, and converges exponentially to
the objective function.

Two important areas of future work are improvements to the Monte Carlo
adjoint solver and applications of the interpolation scheme.
To further improve the accuracy
of the Monte Carlo adjoint solver and reduce its cost, variance-reduction
methods can be used on the Monte Carlo linear solver.  In addition, formulating
the linear system on a different linear basis may significantly reduce the
variance.  For example, we plan to formulate the adjoint equation on a Fourier
basis and analyze its impact on the accuracy of the Monte Carlo solution.
If we can reduce the variance of our Monte Carlo adjoint solver, we can
apply this method to the adjoint of the Navier-Stokes equations.

The second area of future work is in the application of the multi-variate
interpolation method.  We plan to apply this method to the
Navier-Stokes equations,
and compare it to uncertainty quantification methods such as
polynomial chaos and collocation methods.  In addition, we want to use
the interpolation method to augment collocation methods.  In a collocation
method, it is possible that one level of tensor or sparse grid is not
sufficient to obtain accurate results, while the next grid level is too
expensive to calculate.  In this case, one can add as many grid points as
are affordable, and use our interpolation method to obtain the value at the
uncalculated points in the next level grid.  Such a method could
make collocation methods more practical when the dimensionality of the
random space is high.
